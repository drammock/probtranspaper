\subsection{ASR Trained Using Probabilistic Transcriptions}
\label{ssec:asr}

\setlength{\tabcolsep}{0.37cm}
\begin{table*}[t]
\centerline{\begin{tabular}{| c || c | c | c | c | c |}\hline
Language &  Cross-Lingual & Self-training & \multicolumn{3}{ |c| }{CL + PT adaptation}  \\\cline{4-6}
Code & ({\sc CL}) & ({\sc ST}) &  ({\sc PT-adapt}) & \% Rel. redn & \% Rel. redn\\\cline{4-6}
 &&&& \% over {\sc CL} & \% over {\sc ST}\\
\hline
\multicolumn{6}{|l|}{GMM-HMM} \\\hline
yue & 68.40 (68.35) & &  \textbf{57.20 (56.57)} &  16.4** (17.1) & 10.3** (9.2) \\
hun & 68.62 (66.90) & &   \textbf{56.98 (57.26)} & 16.9** (14.3) & 10.2** (9.9) \\
cmn & 71.30 (68.66) & &   \textbf{58.21 (57.85)} &  18.4** (15.7) & 10.3** (9.7) \\
swh & 63.04 (64.73) & &   \textbf{44.31 (48.88)} & 29/6** (24.6) & 24.7** (18.4) \\
\hline\hline
\multicolumn{6}{|l|}{NN-HMM} \\\hline
yue & 66.59 (65.41) & 63.79 (62.46) &&& \\
hun & 66.43 (67.18) & 63.53 (63.50) &   \textbf{56.70 (58.45)} & 14.6** (13.0) & 10.8** (8.0) \\
cmn & 65.77 (64.80) & 64.90* (64.00) &   \textbf{54.07 (53.13)} & 17.8** (18.0) & 16.7** (17.0) \\
swh & 65.30 (65.11) &  58.76** (59.81) &   \textbf{44.73 (48.60)} & 31.5** (25.4) & 23.9** (18.7) \\\hline
\end{tabular}}
\caption{\label{tab:ptresult} PERs on the evaluation and development sets (latter within parentheses) before and after adaptation with PTs.  MAPSSWE significance testing: * means $p\le 0.003$, ** means $p<0.001$.}
\end{table*}

This section demonstrates that PT adaptation improves the
generalization capability of cross-lingual ASR to an unseen target
language.  Adaptation to ASR-derived PTs (self-training) significantly
reduces PER, as has been previously
reported~\cite{vesely2013-semi}. PTs derived from human mismatched
crowdsourcing provide significant further PER reduction.

Table~\ref{tab:ptresult} presents phone error rates (PERs) on the
evaluation (and development) sets for four different languages. The
column titled {\sc CL} lists cross-lingual baseline error rates,
copied from Table~\ref{tab:results}.

In Table~\ref{tab:ptresult}, the column labeled {\sc ST} lists the
PERs of self-trained ASR systems. Self-training was only performed
using NN systems; no self-training of GMMs was performed, because
previous studies~\cite{Huang2013} reported it to be less effective.
Differences between the evaluation set PERs of {\sc ST} and {\sc CL}
systems were tested for statistical significance using the MAPSSWE
test of the {\tt sc\_stats} tool~\cite{Pallet90}.  There are 20
independent statistical comparisons in Table~\ref{tab:ptresult}; the
study-corrected significance level of $0.05/20=0.0025$ was rounded up
to $0.003$ because {\tt sc\_stats} only provides three significant
figures.  The Mandarin {\sc ST} system was judged significantly better
than {\sc CL} at a level of $p=0.003$ (denoted *), and the Swahili
system at a level of $p<0.001$ (denoted **); the Cantonese and
Hungarian {\sc ST} systems were judged to be not significantly better
than {\sc CL}.

The column headed {\sc PT-adapt} in Table~\ref{tab:ptresult} lists
PERs from ASR systems that have been adapted to PTs in the target
language. We observe substantial PER improvements using {\sc PT-adapt}
over {\sc CL} across all four languages. We also find that PT
adaptation consistently outperforms the {\sc ST} systems for all four
languages. The relative reductions in PER compared to both baselines
are listed in the last two columns.  Reductions on the evaluation set
were tested for statistical significance using the MAPSSWE test of the
{\tt sc\_stats} tool.  All differences were found to be statistically
significant at $p<0.001$ (denoted **).  This suggests that adaptation
with PTs is providing more information than that obtained by model
self-training alone. It is also interesting that PER
improvements for Swahili are larger than for the other three
languages. We conjecture this may be partly because Swahili's
orthography is based on the Roman alphabet, unlike the other three
languages. Since the mismatched transcripts also used the Roman
alphabet, the PTs derived from them may more closely resemble the
native Swahili transcriptions (from which the phonetic transcriptions
are derived).

It is also useful to compare the performance of GMM-HMM and NN-HMM
systems.  In the {\sc CL} setting, an ASR trained
using six languages is then applied to an unseen seventh language,
without adaptation; in this setting, the NN consistently outperforms
the GMM.  In the {\sc PT-adapt} setting, either GMMs or NNs are
adapted using PTs in the target language.  PT adaptation improves the
performance of both types of ASR, but the NN does not improve as much
as the GMM.

