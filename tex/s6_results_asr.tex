\subsection{ASR Trained Using Probabilistic Transcriptions}
\label{ssec:asr}

\setlength{\tabcolsep}{0.37cm}
\begin{table*}[t]
\centering
\begin{tabular}{| c || c | c | c | c | c | c | c |}
\hline
Language &  Multilingual & Self-training & \multicolumn{3}{ |c| }{Mult-L + PT adaptation}  \\
\cline{4-6}
Code & ({\sc Mult-L}) & ({\sc ST}) &  ({\sc PT-adapt}) & \% Rel. redn & \% Rel. redn\\
\cline{4-6}
 &&&& \% over {\sc Mult-L} & \% over {\sc ST}\\
\hline
\multicolumn{8}{|l|}{GMM-HMM} \\
\hline
yue & 68.40 (68.35) & &  \textbf{57.20 (56.57)} &  16.4 (17.1) & 10.3 (9.2) \\
hun & 68.62 (66.90) & &   \textbf{56.98 (57.26)} & 16.9 (14.3) & 10.2 (9.9) \\
cmn & 71.30 (68.66) & &   \textbf{58.21 (57.85)} &  18.4 (15.7) & 10.3 (9.7) \\
swh & 63.04 (64.73) & &   \textbf{44.31 (48.88)} & 29.6 (24.6) & 24.7 (18.4) \\
\hline\hline
\multicolumn{8}{|l|}{DNN-HMM} \\
\hline
yue & 66.54 (65.28) & 63.79 (62.46) &&& \\
hun & 66.08 (66.58) & 63.53 (63.50) &   \textbf{xxx (58.4)} & xxx (12.3) & xxx (8.0) \\
cmn & 65.77 (64.80) & 64.90 (64.00) &   \textbf{xxx (53.1)} &  xxx (18.1) & xxx (17.0) \\
swh & 64.75 (65.04) &  58.76 (59.81) &   \textbf{xxx (48.6)} & xxx (25.3) & xxx (18.7) \\
\hline
\end{tabular}
\caption{\label{tab:ptresult} PERs on the evaluation and development sets (latter within parentheses) before and after adaptation with PTs.  Asterisk = Eval set PER significantly lower than the PER of the MULT-L baseline in the same row.}
\end{table*}

This section demonstrates that PT adaptation improves the
generalization capability of multilingual ASR to an unseen target
language.  Adaptation to ASR-derived PTs (self-training) significantly
reduces PER, as has been previously
reported~\cite{vesely2013-semi}. PTs derived from human mismatched
crowdsourcing provide significant further PER reduction.

Table~\ref{tab:ptresult} presents phone error rates (PERs) on the
evaluation (and development) sets for four different languages. {\sc
  Mult-L} corresponds to the multilingual baseline error rates
reproduced from Table~\ref{tbl:results}.  PERs of GMM-HMM systems are
reproduced in rows 4-7; PERs of DNN-HMM systems are reproduced in rows
9-12.

{\sc ST} refers to the NN-HMM multilingual baselines adapted with
automatically generated PTs (ASR self-training). Self-training was
only performed using DNN systems; no self-training of GMMs was
performed.  Asterisk denotes a system whose eval-set PER is
significantly lower than the eval-set PER of the MULT-L baseline in
the same row ($p<0.05$, MAPSSWE test computed using the NIST {\tt
  sc\_stats} tool~\cite{Pallet90}).  {\bf (TODO: RUN THE TESTS, INSERT
  ASTERISKS)}

{\sc PT-adapt} corresponds to PERs from multilingual speech
recognition systems that have been adapted to PTs in the target
language. We observe substantial PER improvements using {\sc PT-adapt}
over {\sc Mult-L} across all four languages. We also find that PT
adaptation consistently outperforms the {\sc ST} systems for all four
languages. The relative reductions in PER compared to both baselines
are listed in the last two columns.  This suggests that
adaptation with PTs is providing more information than that obtained
by model self-training alone. It is also interesting that we obtain
significantly larger PER improvements with PTs for Swahili compared to
the other three languages. We conjecture this may be partly because
Swahili's orthography is based on the Roman alphabet unlike the other
three languages. Since the mismatched transcripts also used the Roman
alphabet, the PTs derived from them may more closely resemble the
native Swahili transcriptions (from which the phonetic transcriptions
are derived).

It is also useful to compare the performance of GMM-HMM systems (rows
4-7 of Table~\ref{tab:ptresult}) with the performance of DNN-HMM
systems (rows 9-12).  In the {\sc MULT-L} setting, an ASR trained
using six languages is then applied to an unseen seventh language,
without adaptation; in this setting, the DNN consistently outperforms
the GMM (the difference is not statistically significant, because we
have only ten minutes per language of training data, but it is
consistent across languages).  In the {\sc PT-adapt} setting, either
GMMs or DNNs are adapted using PTs in the target language.  PT
adaptation improves the performance of both types of ASR, but the DNN
does not improve as much as the GMM.

