\subsection{ASR Trained Using Probabilistic Transcriptions}
\label{ssec:asr}

\setlength{\tabcolsep}{0.37cm}
\begin{table*}[t]
\centering
\begin{tabular}{| c || c | c | c | c | c | c | c |}
\hline
Language &  Multilingual & Self-training & \multicolumn{3}{ |c| }{Mult-L + PT adaptation}  \\
\cline{4-6}
Code & ({\sc Mult-L}) & ({\sc ST}) &  ({\sc PT-adapt}) & \% Rel. redn & \% Rel. redn\\
\cline{4-6}
 &&&& \% over {\sc Mult-L} & \% over {\sc SS}\\
\hline
yue & 68.40 (68.35) & 63.79 (62.46) &  \textbf{57.20 (56.57)} &  16.4 (17.1) & 10.3 (9.2) \\
hun & 68.62 (66.90) & 63.53 (63.50) &   \textbf{56.98 (57.26)} & 16.9 (14.3) & 10.2 (9.9) \\
cmn & 71.30 (68.66) & 64.90 (64.00) &   \textbf{58.21 (57.85)} &  18.4 (15.7) & 10.3 (9.7) \\
swh & 63.04 (64.73) & 58.76 (59.81) &   \textbf{44.31 (48.88)} & 29.6 (24.6) & 24.7 (18.4) \\
\hline
\end{tabular}
\caption{\label{tab:ptresult} PERs on the evaluation and development sets (latter within parentheses) before and after adaptation with PTs.}
\end{table*}

This section demonstrates that PT adaptation improves the
generalization capability of multilingual ASR to an unseen target
language.  Adaptation to ASR-derived PTs (self-training) significantly
reduces PER, as has been previously
reported~\cite{vesely2013-semi}. PTs derived from human mismatched
crowdsourcing provide significant further PER reduction.

Table~\ref{tab:ptresult} presents phone error rates (PERs) on the
evaluation (and development) sets for four different languages. {\sc
  Mult-L} corresponds to the multilingual GMM-HMM baseline error rates
reproduced from Table~\ref{tbl:results}.

{\sc ST} refers to the NN-HMM multilingual baselines adapted with
automatically generated PTs (ASR self-training). We observe a
consistent drop in error rates moving from {\sc Mult-L} to {\sc ST}.

{\sc PT-adapt} corresponds to PERs from the multilingual GMM-HMM
systems adapted to mismatched transcriptions from the target
language. We observe substantial PER improvements using {\sc PT-adapt}
over {\sc Mult-L} across all four languages. We also find that PT
adaptation consistently outperforms the {\sc ST} systems for all four
languages. (The relative reductions in PER compared to both baselines
are listed in the last two columns.) This suggests that adaptation
with PTs is providing more information than that obtained by model
self-training alone. It is also interesting that we obtain
significantly larger PER improvements with PTs for Swahili compared to
the other three languages. We conjecture this may be partly because
Swahili's orthography is based on the Roman alphabet unlike the other
three languages. Since the mismatched transcripts also used the Roman
alphabet, the PTs derived from them may more closely resemble the
native Swahili transcriptions (from which the phonetic transcriptions
are derived).

Table~\ref{tab:nnpt} shows PERs of multilingula NN-HMM systems,
both with and without PT adaptation, as compared to corresponding
GMM-HMM systems.  As shown, an NN-HMM applied to a language not seen
during training outperforms a GMM-HMM applied in the same way.
Adaptation using mismatched crowdsourcing significantly improves both
GMM and NN systems, but using the architecture described in this
paper, does not yet allow NN systems to achieve error rates lower than
those of the corresponding GMMs.

((INSERT A TABLE HERE INCLUDING DNN RESULTS))

