\subsection{ASR Trained Using Probabilistic Transcriptions}
\label{ssec:asr}

\setlength{\tabcolsep}{0.37cm}
\begin{table*}[t]
\centerline{\begin{tabular}{| c || c | c | c | c | c |}\hline
Language &  Multilingual & Self-training & \multicolumn{3}{ |c| }{Mult-L + PT adaptation}  \\\cline{4-6}
Code & ({\sc Mult-L}) & ({\sc ST}) &  ({\sc PT-adapt}) & \% Rel. redn & \% Rel. redn\\\cline{4-6}
 &&&& \% over {\sc Mult-L} & \% over {\sc ST}\\
\hline
\multicolumn{6}{|l|}{GMM-HMM} \\\hline
yue & 66.5 (65.9) & &  \textbf{54.7 (54.3)} &  17.7** (17.6) & 11.3** (9.8) \\
hun & 63.0 (63.0) & &   \textbf{53.0 (53.3)} & 15.9** (15.4) & 9.9** (16.1) \\
cmn & 68.0 (64.0) & &   \textbf{55.1 (54.0)} &  19.0** (15.6) & 9.2** (15.6) \\
swh & 61.5 (63.2) & &   \textbf{43.5 (48.0)} & 29.3** (24.1) & 23.4** (17.4) \\
\hline\hline
\multicolumn{6}{|l|}{NN-HMM} \\\hline
yue & 64.2 (62.6) & 61.7 (60.2) &&& \\
hun & 61.2 (61.9) & 58.8 (63.5) &   \textbf{52.1 (54.1)} & 14.9** (12.6) & 11.4** (14.8) \\
cmn & 61.3 (59.4) & 60.7* (64.0) &   \textbf{51.2 (49.4)} & 16.5** (16.8) & 15.7** (22.8) \\
swh & 62.8 (63.0) &  56.8** (58.1) &   \textbf{43.9 (47.3)} & 30.1** (24.9) & 22.7** (18.6) \\\hline
\end{tabular}}
\caption{\label{tab:ptresult} PERs on the evaluation and development sets (latter within parentheses) before and after adaptation with PTs.  MAPSSWE significance testing: *=$p\le 0.003$, *=$p<0.001$.}
\end{table*}

This section demonstrates that PT adaptation improves the
generalization capability of multilingual ASR to an unseen target
language.  Adaptation to ASR-derived PTs (self-training) significantly
reduces PER, as has been previously
reported~\cite{vesely2013-semi}. PTs derived from human mismatched
crowdsourcing provide significant further PER reduction.

Table~\ref{tab:ptresult} presents phone error rates (PERs) on the
evaluation (and development) sets for four different languages. The
column {\sc Mult-L} lists multilingual baseline error rates.  The
recognizer output transcripts on which these scores are based are
identical to those reported in rows four and five of
Table~\ref{tbl:results}, but the reported PERs are lower, for two
reasons.  First, Table~\ref{tab:results} includes errors in the
recognition of transcribed silences, which are not included as errors
in Table~\ref{tab:ptresult}.  Second, Table~\ref{tab:results}
considers the geminate phonemes \ipa{[a:,i:,u:,n:,k:,S:]} to be
distinct from their non-geminate counterparts \ipa{[a,i,u,n,k,S]},
whereas Table~\ref{tab:ptresult} eliminates geminate distinctions
prior to counting recognition errors.  PERs of GMM-HMM systems are
reproduced in rows 4-7 of Table~\ref{tab:ptresult}; PERs of NN-HMM
systems are reproduced in rows 9-12.

In Table~\ref{tab:ptresult}, the column labeled {\sc ST} lists the
PERs of self-trained ASR systems. Self-training was only performed
using NN systems; no self-training of GMMs was performed.  Differences
between the evaluation set PERs of {\sc ST} and {\sc MULT-L} systems
were tested for statistical significance using the MAPSSWE test of the
{\tt sc\_stats} tool~\cite{Pallet90}.  The Mandarin {\sc ST} system
was judged significantly better than {\sc MULT-L} at a level of
$p=0.003$ (denoted *), and the Swahili system at a level of $p<0.001$
(denoted **); the Cantonese and Hungarian {\sc ST} systems were judged
to be not significantly better than {\sc MULT-L}.

The column headed {\sc PT-adapt} in Table~\ref{tab:ptresult} lists
PERs from ASR systems that have been adapted to PTs in the target
language. We observe substantial PER improvements using {\sc PT-adapt}
over {\sc Mult-L} across all four languages. We also find that PT
adaptation consistently outperforms the {\sc ST} systems for all four
languages. The relative reductions in PER compared to both baselines
are listed in the last two columns.  Reductions on the evaluation set
were tested for statistical significance using the MAPSSWE test of the
{\tt sc\_stats} tool.  All differences were found to be statistically
significant at $p<0.001$ (denoted **).  This suggests that adaptation
with PTs is providing more information than that obtained by model
self-training alone. It is also interesting that we obtain larger PER
improvements with PTs for Swahili compared to the other three
languages. We conjecture this may be partly because Swahili's
orthography is based on the Roman alphabet, unlike the other three
languages. Since the mismatched transcripts also used the Roman
alphabet, the PTs derived from them may more closely resemble the
native Swahili transcriptions (from which the phonetic transcriptions
are derived).

It is also useful to compare the performance of GMM-HMM systems (rows
4-7 of Table~\ref{tab:ptresult}) with the performance of NN-HMM
systems (rows 9-12).  In the {\sc MULT-L} setting, an ASR trained
using six languages is then applied to an unseen seventh language,
without adaptation; in this setting, the NN consistently outperforms
the GMM.  In the {\sc PT-adapt} setting, either GMMs or NNs are
adapted using PTs in the target language.  PT adaptation improves the
performance of both types of ASR, but the NN does not improve as much
as the GMM.

