\subsection{Electrophysiology of Speech Perception}

The human auditory system is sensitive to within-category distinctions
in speech sounds, and such pre-categorical perceptual distinctions may
be lost in transcription tasks, where listeners must filter their
percepts through the limited number of categorical representations
available in their native language orthography.  EEG distribution
coding is a proposed new method that interprets the electrical evoked
potentials of untrained listeners (measured by an
electro-encephalograph or EEG) as a posterior probability distribution
over the phone set of the utterance language
(Fig.~\ref{fig:eeg_paradigm}).  Transcribers, in this scenario, listen
to speech in both their native language and an unfamiliar non-native
target language, while their EEG responses are recorded.  From their
responses to English speech, an English-language EEG phone recognizer
is trained~\cite{Liberto15}.  Misperception probabilities
$\rho(\psi|\phi)$ are then estimated: for each non-native phone
$\phi$, the classifier outputs are interpreted as an estimate of
$\rho(\psi|\phi)$ for all $\phi\in\mathbb{\Phi}$, the native phone
inventory.

\begin{figure}
  \centerline{\includegraphics[width=4in]{../figs/Slide2JSALT.pdf}}
  \caption{EEG responses are recorded while listeners hear speech in
    their native language and an unfamiliar non-native language.  For 
    each listener, a bank of distinctive feature classifiers are trained. 
    Those classifiers are then applied to the EEG responses to non-native
    speech, estimating a listener-language transcription of the
    non-native speech.}
  \label{fig:eeg_paradigm}
\end{figure}
