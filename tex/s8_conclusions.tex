
%%%%%%%%%%%%%%%%%%%%%%%%%%%%%%%%%%
\section{Conclusions}

When a language lacks transcribed speech, other types of information
about the speech signal may be used to train ASR.  This paper proposes
compiling the available information into a probabilistic
transcription: a pmf over possible phone transcriptions of each
waveform.  Three sources of information are discussed: self-training,
mismatched crowdsourcing, and EEG distribution coding.  Experiments
demonstrate that self-training outperforms cross-lingual ASR in two of
the four test languages (Mandarin and Swahili).  Adaptation using
mismatched crowdsourcing outperforms both cross-lingual ASR and
self-training in all four of the test languages.  Auxiliary
information from EEG is used, together with text-based phone language
models, to improve the decoding of transcripts from mismatched
crowdsourcing.  
