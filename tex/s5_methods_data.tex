\subsection{Data}
\label{sec:data}

Speech data were extracted from publicly available podcasts~\cite{SBS}
hosted in 68 different languages.  In order to generate test corpora
(in which it is possible to measure phone error rate), advertisements
were posted at the University of Illinois seeking native speakers
willing to transcribe speech in any of these 68 languages.  Of the ten
transcribers who responded, six people were each able to complete one
hour of speech transcription (the other four dropped out).  One
additional language was transcribed by workers recruited at $I^2R$ in
Singapore, yielding a total of seven languages with native
transcriptions suitable for testing an ASR: Arabic (arb), Cantonese
(yue), Dutch (nld), Hungarian (hun), Mandarin (cmn), Swahili (swh) and
Urdu (urd).

The podcasts were not entirely homogeneous in the target
language and contain utterances interspersed with segments of music
and English. A simple GMM-based language identification system was
developed as a first pass over the podcasts in order to isolate
regions that correspond mostly to the target language. These long
segments were then split into smaller $\approx$ 5-second
segments. This was to enable easy labeling by the native transcribers,
and more importantly to allow for the collection of mismatched
transcriptions that required the speech segments to be short. To
further check that only speech clips in the target language were
retained, the native transcribers were asked to omit any 5-second
clips that contained music, significant amounts of noise, English
speech or speech from multiple speakers. The resulting transcribed
speech clips roughly amounted to 45 minutes of speech in Urdu and 1
hour of speech in the remaining six languages. The orthographic
transcriptions for these clips were then converted into phonemic
transcriptions using language-specific dictionaries and G2P mappings
(these resources are detailed in Section~\ref{sec:mlbaseline}). For
each language, we chose a random 40/10/10 minutes split into training,
development and evaluation sets.  Table~\ref{tab:data} describes the
resulting training, development and evaluation sets.
\begin{table}[t]
\centering
\begin{tabular}{|c||c|c|c|c|c|c|c|}\hline
Speech  & \multicolumn{7}{|c|}{Language Code (ISO 639-3)}\\\hline
(\# phones) & arb & yue & nld & hun & cmn & swh & urd \\ \hline\hline
Train & 32486 & 32693 & 27314 & 29461 & 28571 & 30009 & 21275 \\
Dev & 8208 & 6860 & 6943 & 7873 & 8244 & 7658 & 5808 \\
Eval & 8191 & 8638 & 6582 & 7474 & 7035 & 7441 & 3689 \\\hline
\end{tabular}
\caption{Data statistics for seven podcast languages listing number of phones in the training/development/evaluation sets.}
\label{tab:data}
\end{table}
