In many under-resourced languages it is possible to find text, and it is possible to find speech, but transcribed speech suitable for training automatic speech recognition (ASR) is unavailable.  In the absence of native transcripts, this paper proposes the use of a probabilistic transcript: a probability mass function over possible phonetic transcripts of the waveform.  Three sources of probabilistic transcripts are demonstrated.  First, self-training is a well-established semi-supervised learning technique, in which a cross-lingual ASR first labels unlabeled speech, and is then adapted using the same labels.  Second, mismatched crowdsourcing is a recent technique in which non-speakers of the language are asked to write what they hear, and their nonsense transcripts are decoded using noisy channel models of second-language speech perception.  Third, EEG distribution coding is a new technique in which non-speakers of the language listen to it, and their electrocortical response signals are interpreted to indicate probabilities.  ASR was trained in four languages without native transcripts.  Adaptation using mismatched crowdsourcing significantly outperformed self-training, and both significantly outperformed a cross-lingual baseline.  EEG distribution coding and text-derived phone language models were both shown to improve the quality of probabilistic transcripts derived from mismatched crowdsourcing.
