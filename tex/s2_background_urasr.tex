\subsection{Existing Approaches to ASR in Under-Resourced Languages}

Krauwer~\cite{Krauwer2003} defined an under-resourced language to be
one that lacks one or more of: stable orthography, significant
presence on the internet, linguistic expertise, monolingual tagged
corpora, bilingual electronic dictionaries, transcribed speech,
pronunciation dictionaries, or other similar electronic resources.
Berment~\cite{Berment2004} defined a rubric for tabulating the
resources available in any given language, and proposed that a
language should be called ``under-resourced'' if it scored lower than
10.0/20.0 on the proposed rubric.  By these standards, technology
methods for under-resourced languages are most often demonstrated on
languages that are not really under-resourced: for example, ASR may be
trained without transcribed speech, but the quality of the resulting
ASR can only be scientifically proven by measuring its phone error
rate (PER) or word error rate (WER) using transcribed speech.  The
intention, in most cases, is to create methods that can later be
ported to languages that truly lack resources.

The International Phonetic Alphabet (IPA~\cite{ipa1993}) is a set of
symbols representing speech sounds (phones) defined by the principle
that, if two phones are used in any language to make meaningful
linguistic contrasts (i.e., they represent distinct phonemes), then
those phones should have distinct symbolic representations in the IPA.
This makes the IPA a natural choice for transcriptions used to train
cross-language ASR systems, and indeed ASR in a new language can be 
rapidly deployed using acoustic models trained to represent every 
distinct symbol in the IPA~\cite{Schultz2001}.
However, because IPA symbols are defined phonemically, there is no
guarantee of cross-language equivalence in the acoustic properties of
the phones they represent. This problem arises even between dialects of
the same language: a monolingual Gaussian mixture model (GMM) trained on
five hours of Levantine Arabic can be improved by adding ten hours of
Standard Arabic data, but only if the log likelihood of cross-dialect
data is scaled by 0.02~\cite{Huang2012}.

Better cross-language transfer of acoustic models can be achieved, but only
by using structured transfer learning methods, including neural networks
(NN) and subspace Gaussian mixture models (SGMM).
NN transfer learning can be categorized as tandem, bottleneck,
pre-training, phone mapping, and multi-softmax methods.  In a tandem
system, outputs of the NN are Gaussianized, and used as features whose
likelihood is computed with a GMM~\cite{Hermansky2000}; in a
bottleneck system, features are extracted from a hidden layer rather
than the output layer. Both tandem~\cite{Stolcke2006} and
bottleneck~\cite{Vesely2012} features trained on other languages can
be combined with GMMs trained on the target language in order to
improve word error rate (WER).

A hybrid ASR is a system in which the NN terminates in a softmax
layer, whose outputs are interpreted as phone~\cite{Morgan95} or
senone~\cite{Dahl2012} probabilities.  Knowledge of the target
language phone inventory is necessary to train a hybrid ASR, but it is
possible to reduce WER by first pre-training the NN hidden layers with
multilingual data~\cite{Huang2013,Swietojanski2012}.  A hybrid ASR can be
constructed using very little in-language speech data by adding a
single phone-mapping layer to the output of the multilingual NN; the
phone mapping layer can be trained using a small amount of in-language
speech data~\cite{Sim2008}, even if context-dependent senones are
mapped instead of phones~\cite{Do2012}.  A multi-softmax system
integrates phone mapping into the original training procedure, by
training a network with several different language-dependent softmax
layers, each of which is the linear transform of a multilingual shared
hidden layer.  Multi-softmax systems have reduced WER in
tandem~\cite{Scanzio2008}, bottleneck~\cite{Vesely2012}, and
hybrid~\cite{Huang2013} ASR.

SGMM transfer learning uses language-dependent GMMs, each of which is
the linear interpolation of language-independent mean and variance
vectors.  SGMM can be combined with other methods for further
improvement, e.g., 16\% relative WER reduction was achieved in a Tamil
ASR by combining SGMM with an acoustic data normalization
technique~\cite{Mohan2014}, and further reductions were obtained in
Afrikaans by using bottleneck features in an SGMM~\cite{Imseng2014}.

%% Self-training is one of the three methods highlighted in the 
%% introduction. Doesn't it deserve its own sub-section?
Self-training is a class of semi-supervised learning techniques in
which ASR is first trained on labeled corpora in other languages, then
used to label data in the target language.  Self-labeled data in the
target language is then used to train or adapt the
ASR~\cite{Loof2009,Cetin2008}.  Self-training is most useful when the
in-language training data are first filtered, to exclude frames with
confidence below a threshold.  The posterior probability computed from
the ASR lattice is a useful confidence score~\cite{vesely2013-semi},
but it is also possible to learn an improved confidence score by
combining multiple sources of information~\cite{Vu2011b}.

Under-resourced languages often lack any pronunciation dictionary.  It
is possible to train a stable grapheme-to-phoneme transducer using a
dictionary with 15,000 entries, and in some languages a dictionary of
this size can be mined from sources such as
Wiktionary~\cite{Schlippe2014}.  In languages without any dictionary
of this size, it may be possible to approximate pronunciation by
treating each orthographic character as an acoustic 
model~\cite{Kanthak2002,Charoenpornsawat06,Gizaw2008,Le2009}.
Even an ambiguous G2P can often be disambiguated by the use of
context-dependent graphemic models~\cite{Kanthak2002}; if the number
of trigraphemes gets too large, acoustic models can be interpolated
within an eigentrigrapheme space~\cite{Ko2014}.  Optimal WER in
Standard Arabic was achieved by using phoneme-based pronunciations for
the most frequent 500 words, and grapheme-based pronunciations for all
less frequent words~\cite{Elmahdy2012}.  In Amharic, optimal WER was
achieved using a morpheme-based language model, combined with a hybrid
acoustic model space including both triphones and context-dependent
sylabic units~\cite{Tachbelie2014}.  In Hindi, optimal WER was
achieved using a one-to-one character-based grapheme-to-phoneme (G2P)
transducer (essentially a grapheme-based acoustic model), modified by
a very small set (3) of surface phonological
rules~\cite{Jyothi2015interspeech_hindi}.  The three rules were
proposed based on phonological descriptions of Hindi, then applied or
discarded in response to application probabilities learned using a
very small (200-word) pronunciation dictionary.

